\subsection{$R_{eq}$ and time constant $\tau$}
Next up, we need to determine the equivalent resistance, $R_eq$, as seen from the capacitor terminals to find the time constant associated with this circuit, $\tau = R_{eq}C$, so that one can analyse, afterwards, the natural and the forced solution of the circuit. \\

To do that, one can start by "shutting down" the independent voltage source of the circuit, $V_s$, by making it $V_s = 0$, i.e., replacing it by a short circuit. Afterward, as we want to calculate $R_{eq}$ as seen by the terminals of the capacitor, one should now substitute the capacitor by a voltage source with value $V_x = V_6 - V_8$, where one can use the values obtained in section \ref{sec:1st} for $V_6$ and $V_8$. This whole process is adopted because the value $R_{eq}$ which we are seeking is nothing more than the Thévenin Resistance, $R_{th}$, of the circuit as seen by the terminals of the capacitor. \\

Rewriting the system of equations with this new constraints (knowing the value for $V_x$ and imposing $V_s = 0$), one can write the system as:

\begin{equation}
    \begin{bmatrix}
     1 &  0      &  0 &    -1  &     0      &  0  &  0    &  0\\
     -G_1 & G_1+G_2+G_3 & -G_2  & 0   &  -G_3       &  0  &  0    &  0\\
     0   & -G_2-K_b    & G_2  & 0   &   K_b       &  0  &  0    &  0\\
     0   & 0        & 0   & 1 & 0       &  0  & 0   & 0\\
     0   & -G_3+K_b      &  0 &  -G_4  &   G_3+G_4-K_b & 0 &  -G_7    & G_7\\
     0   & 0       & 0  &  0    & 0     & 1  & 0     & -1\\
     0   & 0        & 0  & -G_6   &  0         & 0   & G_6+G_7 & -G_7\\
     0   & 0        & 0  &  -K_dG_6    &  1         & 0   & K_dG_6     & -1
    \end{bmatrix} 
    \begin{bmatrix}
        V_1\\
        V_2\\
        V_3\\
        V_4\\
        V_5\\
        V_6\\
        V_7\\
        V_8
    \end{bmatrix}
    = 
    \begin{bmatrix}
        0\\
        0\\
        0\\
        0\\
        0\\
        V_x\\
        0\\
        0
    \end{bmatrix}
\label{eqnodos}
\end{equation}

Which take us to this result:

\begin{equation}
    \begin{bmatrix}
        V_1\\
        V_2\\
        V_3\\
        V_4\\
        V_5\\
        V_6\\
        V_7\\
        V_8
    \end{bmatrix}
    = 
    \begin{bmatrix}
        0\\
        0\\
        0\\
        0\\
        0\\
        8.40363\\
        0\\
        0
    \end{bmatrix}V
\end{equation}

Now that one obtained the values for every node voltage, and having in mind that $I_x$ can be written as $I_x = -(I_b+I_5)$:

\begin{equation}
\begin{cases}
    I_b = K_b(V_2 - V_5) = 0mA\\
    I_5 = (V_6 - V_8)G_5 \approx -2.80 mA
\end{cases}
    \Longrightarrow I_x = -I_5 = 2.80 mA
\end{equation}

Now that one obtained the values of $V_x$ and $I_x$, one can finally obtain the value for $R_eq$, which can be written as:

\begin{equation}
    R_{eq} = \frac{V_x}{I_x} \approx 3.0014 k\Omega
\end{equation}

Finally, as for RC circuits, the time constant can be expressed as:

\begin{equation}
    \tau = R_{eq}
\end{equation}

one can obtain that for this specific circuit, $tau$ can be written as:

\begin{equation}
    \tau \approx 3.0516 ms
\end{equation}