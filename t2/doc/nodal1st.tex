\section{Theoretical Analysis}
\label{sec:theo}
\subsection{Nodal Analysis at $t<0$}
\label{sec:1st}
First of all, one can start by applying the nodal analysis method at $t<0$ to find the voltages of all nodes and, afterwards, using Ohm's law, the currents that flow through each branch of the circuit. Note that the labels used to refer to each node and branch are exactly the same as presented in Fig. \ref{fig:bigscheme}.

So, the next step is to write linearly indepedent equations that will allow us to get the voltages at each node. Note that for $t<0$, we will have $u(t) = 0 \Longrightarrow I_c = 0$ and so, by applying the condition associated with the independent voltage source, one will have a constant value for $v_s(t)$, such that $v_s (t) = V_s$ for $t<0$ (as it can be seen in the expression for the independent voltage source in Fig. \ref{fig:bigscheme}.

\begin{equation}
    \begin{cases}
        V_1 - V_4 = V_s \hspace{15px} \text{(independent voltage source)}\\
        (V_2-V_1)G_1 + (V_2-V_3)G_2 + (V_2-V_5)G_3 = 0 \hspace{15px}\text{(node 2)}\\
        (V_3 - V_2)G_2 + (V_5 - V_2)K_b = 0 \hspace{15px}\text{(node 3)}\\ 
        V_4 = 0 \hspace{15px}\text{(node 4 assigned to GND)}\\
        (V_5-V_2)G_3 + (V_5-V_4)G_4 + (V_5-V_6)G_5 + (V_8-V_7)G_7 = 0 \hspace{15px}\text{(supernodal equation)}\\
        (V_6-V_5)G_5 + (V_2-V_5)K_b = 0 \hspace{15px}\text{(node 6)}\\
        (V_7-V_4)G_6 + (V_7-V_8)G_7 = 0 \hspace{15px}\text{(node 7)}\\
        V_5 - V_8 = (V_7-V_4)K_dG_6 \hspace{15px}\text{(restriction equation involving dependent voltage source)}
    \end{cases}
\end{equation}

Note that the last equation of this system, the restriction equation involving the current-controlled voltage source, urges as the result of the combination of two equations. As shown in Fig. \ref{fig:bigscheme}, one can write $V_d$ as:

\begin{equation}
    V_d = K_d I_d
\end{equation}

and also:

\begin{equation}
    V_d = V_5 - V_8
\end{equation}

Because $I_d$ is the current which flows through the resistor $R_6$ (one could write an equivalent equation by using the resistor $R_7$, as the current flowing through it is exactly the same), one can easily write, using Ohm's Law:

\begin{equation}
    I_d = (V_7 - V_4)G_6
\end{equation}

Combining these 3 equations, one can obtain the restriction equation used in the previous system:

\begin{equation}
    V_d = V_5 - V_8 = (V_7 - V_4)G_6 K_d
\end{equation}

Converting this system of equations to its matricial form, one can obtain:

\begin{equation}
    \begin{bmatrix}
     1 &  0      &  0 &    -1  &     0      &  0  &  0    &  0\\
     -G_1 & G_1+G_2+G_3 & -G_2  & 0   &  -G_3       &  0  &  0    &  0\\
     0   & -G_2-K_b    & G_2  & 0   &   K_b       &  0  &  0    &  0\\
     0   & 0        & 0   & 1 & 0       &  0  & 0   & 0\\
     0   & -G_3      &  0 &  -G_4  &   G_3+G_4+G_5 & -G_5 &  -G_7    & G_7\\
     0   & K_b       & 0  &  0    & -G_5-K_b     & G_5  & 0     & 0\\
     0   & 0        & 0  & -G_6   &  0         & 0   & G_6+G_7 & -G_7\\
     0   & 0        & 0  &  -K_dG_6    &  1         & 0   & K_dG_6     & -1
    \end{bmatrix} 
    \begin{bmatrix}
        V_1\\
        V_2\\
        V_3\\
        V_4\\
        V_5\\
        V_6\\
        V_7\\
        V_8
    \end{bmatrix}
    = 
    \begin{bmatrix}
        V_s\\
        0\\
        0\\
        0\\
        0\\
        0\\
        0\\
        0
    \end{bmatrix}
\label{eqnodos2}
\end{equation}

The matrix system \ref{eqnodos2} is composed of a very sparse matrix, thus we also advise for using correspondingly adapted sparse matrix solvers if possible, but this is already out of the scope of this discipline and laboratory. Getting back to the point, the solution to the matrix system is,

%V1 & 5.038475019720000E+00  \\ \hline
V2 & 4.758223649291032E+00 \\ \hline
V3 & 4.170534768621250E+00 \\ \hline
V5 & 4.798515115188197E+00 \\ \hline
V6 & 5.646841168887640E+00  \\ \hline
V7 & -1.834523049106916E+00 \\ \hline
V8 & -2.756787388808905E+00 \\ \hline


Finally, to find the currents in the circuit, one can apply Ohm's law as one knows all the resistor values and the voltages in all of the circuit's nodes (presented in \eqref{eqsol}. \\

Labeling the positive and negative terminals of each component as inserted in \textit{Ngspice}, and as it can be seen in Fig. something, one can obtain

\begin{table}[H]
    \begin{minipage}{.5\textwidth}
      \begin{equation} \Vec{b} = \begin{bmatrix}  5.0385\\  4.7582\\  4.1705\\      -0\\  4.7985\\  5.6468\\ -1.8345\\ -2.7568 \end{bmatrix} \label{eqsol} \end{equation}
    \end{minipage}
    \begin{minipage}{.5\textwidth}
      \centering
      \begin{tabular}{c|c}
        \hline
          Branch &  Current (A) \\
          \hline
          I(R1) & -0.000269\\
I(R2) & 0.000283\\
I(R3) & -0.000013\\
I(R4) & -0.001172\\
I(R5) & -0.000283\\
I(R6) & 0.000902\\
I(R7) & 0.000902\\
I(Vs) & -0.000269\\
I(Vd) & -0.000902\\
Id  & 0.000902\\
Ib & -0.000283\\
Ic & 0.000000\\

          \hline
      \end{tabular}
    \end{minipage}
    \caption{Theoretical solution for voltages for all nodes and current for all branches for $t<0$.}
    \label{tab:current}
\end{table}

