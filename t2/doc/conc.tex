\section{Conclusion}
The goal of this laboratory assignment was to analyse a RC circuit with sinusoidal excitation. This circuit full analysis was made both theoretically (with the computational help of \textit{Octave}) and computationally, simulating the circuit using \textit{Ngspice}.\\

The comparison of the results obtained from both methods has been made throughout the report, and it was summarized on section \ref{sec:sidebyside} where all the values and plots obtained were compared. \\

Unsurprisingly, the results obtained with both methods were pretty consistent and similar to each other. The values obtained for the nodes voltages and the branches currents in $t<0$ by the simulation matched every digit of the values obtained theoretically (the differences seen are product of approximations done by Octave). In addition, the plots made on both methods to analyse the circuit at $t \geq 0$ were very similar, except the plots obtained to the forced solution on node 6, where the plot obtained by \textit{Ngspice} had a small offset on the beginning of the interval plotted due to an initial transient state, as the circuit doesn't \textit{really} start in equilibrium.\\

Therefore, and concluding, as all the results and \textit{plots} obtained were consistent with each other, it can be stated that the goals for this laboratory were successfully achieved.