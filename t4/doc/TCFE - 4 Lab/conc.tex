\section{Conclusion}
In this laboratory assignment, as presented above, we were proposed to build and study the behaviour of an audio amplifier circuit. To build it, one considered two different regions: a gain stage and an output stage with different objectives. \\

The values of the multiple components on the circuit were adjusted in order to increase the merit score which evaluates the efficiency of the circuit and its cost. To increase the efficiency of the circuit, we tried to maximize the amplifying bandwidth and to minimize the lower cutoff frequency while trying to keep a low total cost for the circuit. \\

To evaluate how good the built amplifier was, we recurred to the merit score presented in Eq. \eqref{score}.  Overall, one can consider the results obtained in the simulation analysis (section 2) as very satisfactory. The results given by the simulation analysis show a bandwidth of $\sim 10^6$ and a lower cutoff frequency $\sim 10$. \\

Comparing the results obtained in the simulation and the theoretical analysis, one can conclude that the general shape and values are similar although some small differences appear. The ones in the gain and in the input and output impedance are the more significant but those may come from the matrix calculations and the approximation models used for the transistors that are the only non linear components in this circuit, divisions by 0 also may appear because the analysis is done in a very wide range of values, from 1 Hz to $10^8$ Hz. These deviations make up for a slightly bigger merit in the theoretical analysis but the one that is the closest to reality is the simulation analysis so the final merit for the circuit is 2790.17.\\

Therefore, and having explained the differences registered between the theoretical and the simulation analysis, it can be stated that the goals for this laboratory assignment were successfully achieved.