\section{Conclusion}
In this laboratory assignment, as presented above, we were proposed to design and implement a band-pass filter using an operational amplifier (OP-AMP). To build it, one considered three different regions: a high-pass filter stage, an amplifying stage (where was the OP-AMP) and a low-pass filter stage. \\

The values of the multiple components on the circuit were adjusted in order to increase the merit score which evaluates the efficiency of the circuit and its cost. To increase the efficiency of the circuit, one tried to increase the voltage gain (to a value around $40dB$) and the central frequency of the output signal (around $1kHz$) while trying to keep a low value for the total cost of the circuit. \\

To evaluate how good the built amplifier was, we recurred to the merit score presented in Eq. \eqref{merit}. Overall, one can consider the results obtained in the simulation analysis (section 2) as very satisfactory. The results given by the simulation analysis show a voltage gain around $35.5 dB$ and a central frequency around $991Hz$ - satisfactory values having in mind that we persecuted a gain of $40dB$ and a central frequency of $1000Hz$.\\

Comparing the results obtained in the simulation and the theoretical analysis, (...)\\

Therefore, and having explained the differences registered between the theoretical and the simulation analysis, it can be stated that the goals for this laboratory assignment were successfully achieved.