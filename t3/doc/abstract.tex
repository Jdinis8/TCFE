\begin{abstract}{english}
    In this report, one analyses how AC/DC converters work with a proposal of a circuit made up of an envelope detector circuit (that converted the AC current into DC and smoothed it) and a voltage regulator circuit (which limited the voltage) that was built to convert an AC current input (with 230V and frequency of 50Hz) into a DC current output with a voltage of 12V. This analysis was made both theoretically and computationally by using \textit{Ngspice} to simulate the circuit (this time the analysis was essentialy made through the \textit{Ngspice} simulation). The results obtained were very satisfactory, in the simulation part. The average output voltage was almost exactly 12V (displacement $\sim 10^{-7}V$), the ripples $\sim 10^{-4}V$ and the merit close to 1000. Relatively to the theoretical analysis, the results obtained weren't as good as in the simulation due to the diode theoretical model used and presented in the classes.
\end{abstract}

\begin{abstract}{portuguese}
    Neste relatório, foi proposta a utilização de um conversor AC/DC com o principal objetivo de converter um \textit{input} de corrente AC (um sinal sinusoidal de frequência 50Hz e com uma tensão associada de 230V) num \textit{output} de corrente DC (com uma tensão nos terminais de saída de 12V). Para tal, o conversor é constituído de um \textit{envelope detector} cuja principal função é converter o sinal de alternado em contínuo e suavizar o mesmo e de um \textit{voltage limiter}, que teve como função limitar superiormente a tensão nos terminais de saída do circuito. Os resultados obtidos foram bastante satisfatórios, na parte da simulação. A voltagem de saída média foi virtualmente 12V (desvio $\sim 10^{-7}V$), o \textit{ripple} $\sim 10^{-4}V$ e o mérito perto de 1000. Relativamente à análise teórica, os resultados obtidos não foram tão bons quanto na simulação, devido ao modelo teórico utilizado para descrever o díodo, apresentado nas aulas.
\end{abstract}