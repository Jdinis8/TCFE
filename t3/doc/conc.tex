\section{Conclusion}
In this laboratory assignment, as presented above, we were proposed to build and study the behaviour of an AC/DC converter. To build it, one used an envelope detector circuit (composed by a full-wave rectifier which turns the AC input into a DC output and a capacitor which smooths the oscillating current) and a voltage regulator circuit (that uses a resistor and multiple diodes which limit the output voltage to the desired value - 12V).\\

To evaluate how good the built converter was, we recurred the merit score presented in \eqref{score}. The results obtained in the simulation analysis (section 2) were very satisfactory except the stabilization time of the circuit, but that wasn't taken into account in the score obtained for the circuit although we had in mind that the circuit shouldn't have a very long stabilization time. The results given by the simulation analysis show an output signal with an average value displaced $\sim 10^{-7}$V from the desired value of 12V and with ripples $\sim 10^{-4}$V. The output voltage obtained is thus almost perfectly equal to the desired output voltage for this converter, which shows that the goal of this simulation was successfully achieved.\\

Comparing the simulation and the theoretical results one can notice a significant difference between the plots and the results obtained with both as anticipated on section 2. In fact, this phenomenon can be justified by the multiple approximations made in the theoretical analysis, namely on the diode model used. In the analysis made using \textit{Octave}, one used the diode models presented in class which neglect the non-linear behaviour of these components which ends up introducing discrepancies between the analysis methods used.\\

However, it is also important to notice that the results obtained on the theoretical analysis were affected of very small ripples, not however as small as in the simulation. To get a better approximation of the merit obtained in the simulation part, we proposed a corrected merit as well, which assumes that the average of the output voltage is exactly $12V$.\\

Therefore, and having explained the differences registered between the theoretical and the simulation analysis, it can be stated that the goals for this laboratory assignment were successfully achieved.