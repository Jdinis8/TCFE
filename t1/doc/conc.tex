\section{Conclusion}
The goal of this assignment was to analyse the circuit using different methods. Theoretically, they were used nodal and mesh methods, that result from the application of Kirchhoff's Laws (current and voltage, respectively). It was also used \textit{Ngspice}, a program that allowed us to simulate the circuit and to obtain the different currents and voltages in it, although this program also uses the theoretical methods used and Kirchhoff's laws in its bases. \\

The mesh analysis provides directly the fictional current, being easy to infer any current on the circuit. On the other hand, nodal analysis allows to determine the voltage in every circuit node and one can confirm that the methods are equivalent by comparing the voltage values $V_c$ and $V_b$ from each one and the value of $V_c$. Doing that, one can prove that the values were coincident in all the provided digits and $V_b$ only diverges from one another in the fifth non-zero digit ( $0.0025\%$ ). Furthermore, another pair of values obtained in both methods is $I_a$ and $I_c$ which one can verify that match at least 5 decimal places, as well.

Being both theoretical methods used very consistent with each other, we already had expected values for each current and voltage that one would obtain from Ngspice's output. Unsurprisingly, those values matched with the obtained values from Ngspice with almost every shown digit being the same as in the first two methods. As said before, the small differences registered in these values may be justified with errors of numeric nature such as in matrix solving algorithms (that for this problem in question could involve inverting big matrices, which can introduce some margin of error) or just the \textit{machine epsilon} of the program(s).

That being said, it was nothing more than expected since the voltages and  currents in this circuit are all constant and the relationship between every component is linear so there was no reason for the simulation to not match the theoretical results. Therefore, and concluding, it can be stated that the goals for this laboratory were successfully achieved.