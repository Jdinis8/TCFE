\section{Simulation Analysis}

To be sure of the values obtained in \ref{sec:theo}, we made use of Ngspice, a more stable version of the Berkeley SPICE, which is an "open source spice simulator for electric and electronic circuits", as stated in \cite{ngsite}. The code used was the following, \\

{\itshape
TCFELab1 \par
.options savecurrents \\\par

Vin 1 4 5.03847501972 \par
Id 0 6 1.01674167773m \par

$V_{dumb}$ CFP1 7 0 \\\par

R1 2 1 1.03994439216k \par
R2 3 2 2.07923431764k \par
R3 2 5 3.06168544529k \par
R4 4 5 4.09516986362k \par
R5 5 6 3.00136467001k \par
R6 4 CFP1 2.03324628446k \par
R7 7 0 1.02216788331k \\\par

Gb 6 3 2 5 7.01505323139m \par
Hc 5 0 $V_{dumb}$ 8.37372457746k \\\par

.op\par
.control\par
    run\par
    print all\par
.endc\par
.end
}

\vspace{10px}
The first line of this code is the title of the simulation itself, which is followed by the option to save currents, a command that allows the program to, when printing all the variables, also print the currents flowing in every component of the circuit. Then, \textit{Vin} and \textit{Id} are the $V_a$, $I_d$ corresponding to what is shown in Fig. \ref{fig:nodal_scheme}. $V_{dumb}$ is a dumb voltage source that was introduced for a specific reason that will be later explained. In this simulation, we chose node 8 to be GND, which, by Ngspice standards, has to be named node 0.\\

After this, all the resistances were declared in the form \textit{R<name> <n+> <n-> <value>}, where n+ and n- mean the positive and negative nodes of the electronic component, according to the schematics shown in Fig.\ref{fig:polos}. \\

\begin{figure}[h]
    \centering
    \includegraphics[width = 0.7\linewidth]{esquemacpolos.pdf}
        \caption{\textit{Representation of the convention used to the Ngspice simulation. In the figure above, one can see the conventioned positive and negative terminal for each branch}}
    \label{fig:polos}
\end{figure}

Afterwards, $Gb$ and $Hc$ stand for the voltage-controlled current source $I_b$ and current-controlled voltage source $V_c$ respectively. The first two parameters received by $Gb$ are its positive and negative nodes as expected and then the two nodes where it should calculate the corresponding potential. For $Hc$, the third parameter is $V_{dumb}$ which, as promised, is a voltage source that was introduced for the single purpose that current-controlled voltage sources in \textit{Ngspice} take as a third parameter only voltage sources through which the current in question is flowing through. Thus, because the purpose is to retrieve $I_c$, a dumb voltage source was added after $R_6$, creating a new node that was called CFTP1, which is where $R_6$ then connects. To not alter the circuit at all, the actual induced voltage by $V_{dumb}$ is 0V, thus CFTP1 and node 7 are in short circuit, i.e., the circuit does not "see" $V_{dumb}$. At last, inside \textit{control}, the simulation is ran and everything is printed out, giving the following output:\\\par


\begin{table}[H]
\setlength{\tabcolsep}{10pt}
\renewcommand{\arraystretch}{1.1}
\centering
    \begin{tabular}{||c|c||}
    \hline
    Name & Current/Voltage [A/V]\\
    \hline
    @gb[i] & -2.82647e-04\\ \hline
@id[current] & 1.016742e-03\\ \hline
@r1[i] & -2.69487e-04\\ \hline
@r2[i] & -2.82647e-04\\ \hline
@r3[i] & -1.31599e-05\\ \hline
@r4[i] & -1.17175e-03\\ \hline
@r5[i] & -1.29939e-03\\ \hline
@r6[i] & 9.022631e-04\\ \hline
@r7[i] & 9.022631e-04\\ \hline
v(1) & 7.795262e+00\\ \hline
v(2) & 7.515011e+00\\ \hline
v(3) & 6.927322e+00\\ \hline
v(4) & 2.756787e+00\\ \hline
v(5) & 7.555303e+00\\ \hline
v(6) & 1.145524e+01\\ \hline
v(7) & 9.222643e-01\\ \hline
cfp1 & 9.222643e-01\\ \hline

    \end{tabular}
\vspace{0.2 cm}
\caption{Values for several variables declared in the .net file.}
\label{max}
\end{table}


\vspace{10px}

The values obtained were expected in some way given the results obtained in the theoretical analysis and that were presented in the last section. The reader can check the results knowing that $V_b = V_2 -V_5$ and $V_c = V_5$.\\

However, due to \textit{machine epsilons} and related things, some decimal places might be a bit different. The main cause of this is that Ngspice also uses, whenever possible, nodal analysis. Thus, for solving matrix \ref{eqnodos}, one has to be aware that inverting a 9x9 or 7x7 matrix will introduce a significant margin of error, which explains a possible observable difference in the results obtained, specially between this simulation and the mesh analysis.
